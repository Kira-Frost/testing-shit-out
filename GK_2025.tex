\documentclass[12pt]{amsart}
\usepackage{amsmath, amssymb, amsthm}
\usepackage{xcolor}
%%%%%
\newcommand{\D}{\mathcal{D}}
\newcommand{\x}{x}
\newcommand{\norm}[1]{|#1|}
\newcommand{\dnorm}[2]{\|#1\|_{#2}}
\newcommand{\hnorm}[2]{\dnorm{#1}{H^{#2}(\un\Omega)}}
\newcommand{\comment}[1]{}
%%%%%


\newtheorem{theorem}{Theorem}[section]
\newtheorem{problem}[theorem]{Problem}
\newtheorem{lemma}[theorem]{Lemma}
\newtheorem{proposition}[theorem]{Proposition}
\newtheorem{corol}[theorem]{Corollary}
\newtheorem{definition}[theorem]{Definition}

\theoremstyle{definition}
\newtheorem{remark}[theorem]{Remark}
\newtheorem{remarks}[theorem]{Remarks}
\newtheorem{example}[theorem]{Example}


\newcommand{\R}{\mathbb{R}}
\newcommand{\N}{\mathbb{N}}
\newcommand{\C}{\mathbb{C}}
\newcommand{\Cl}{\R_{0,n}}
\newcommand{\del}{\delta}
\newcommand{\inci}{\subseteq}
\newcommand{\ba}{\overline}
\newcommand{\Om}{\Omega}
\newcommand{\un}{\underline}



\begin{document}

\title[Dirac operators with distributional coefficients]{Very weak solutions of Dirac operators with distributional coefficients }


\author[N. Gomes]{N. Gomes}
\address{(NG) CIDMA - Center for Research and Development in Mathematics and Applications, \newline Universidade de Cabo Verde \newline Campus 
  \newline Praia, Cabo Verde}
\email{}

\author[K. Morozov]{K. Morozov}
\address{(KM) CIDMA - Center for Research and Development in Mathematics and Applications, \newline Department of Mathematics, University of Aveiro \newline Campus Universit\'ario de Santiago
  \newline 3810-193 Aveiro, Portugal}
\email{}

%
%
\subjclass[2010]{Primary: 30G35; Secondary: }
%
%\date{}
\keywords{Dirac operator, very weak solution}
%%%%%%%%%%%%%





\begin{abstract}
In this paper we will discuss the possibility to study very weak solutions of Dirac operators with distributional coefficients. To this end we need to adapt classic methods like CK-extension from hypercomplex analysis to this case and combine it with techniques developed by M. Ruzhansky and his collaborators. 
\end{abstract}

\maketitle



\section{Introduction}



Let us  consider the following Cauchy problem:
\begin{equation}
  \begin{cases}
\partial_{x_0} u(x_0+\un{x}) + a(x_0)D u(x_0+\un{x}) = 0, & \text{with } x \in (0, T]\times\Omega, \\
u(0,\un{x}) = u_0(\un{x}),\;\; \un{x}\in \Omega,
\end{cases}
\label{CP}  
\end{equation}
where $D := \sum_{k=1}^n e_k \partial_{x_k}$ is the Dirac operator, and $e_k$ are elements of a Clifford algebra satisfying $e_k e_j + e_j e_k = -2\delta_{kj}$.
%%%%%%%%%%%%% Theorem 1 and Theorem 2 %%%%%%%%%%%%%%%%

\begin{theorem}
\label{th1.1}
Let \(x_0 > 0\). Let \(a \in \mathrm{Lip}([0, T])\) with \(a(x_0) \geq a_0 > 0\). Given \(s \in \mathbb{R}\), if the initial Cauchy data \(u_0\) are in \(H^{s}(\Omega)\), then there exists the unique solution of the homogeneous Cauchy problem~(\ref{CP}) (when \(f \equiv 0\)) in the space \(C([0, T], H^{s})\), satisfying the following inequality for all values of \(x_0 \in [0, T]\):
    \begin{equation*}
    \label{eq:1.5}
          \| u(t, \cdot) \|_{H^s(\Omega)}^2  \leq C \| u_0 \|_{H^s(\Omega)}^2 .
    \end{equation*}
\end{theorem}

\begin{theorem}
\label{th1.2}
Let \(x_0 > 0\). Then we have:
Let \(a \in \mathrm{Lip}([0, T])\) with \(a(x_0) \geq a_0 > 0\). Given \(s \in \mathbb{R}\), if \(f \in C([0, T], H^{s}(\Omega))\) and the initial Cauchy data \(u_0\) are in \(H^s\), then there exists a unique solution of~(\ref{CP}) in the space \(C([0, T], H^s(\Omega))\), satisfying the following inequality for all values of \(x_0 \in [0, T]\):
    \begin{equation}
    \label{eq:1.6}
    \| u(t, \cdot) \|_{H^s(\Omega)}^2  \leq C\left(\| u_0 \|_{H^s(\Omega)}^2+\| f\|_{C([0, T], H^{s}(\Omega))}^2\right).
   \end{equation}
\end{theorem}

\section{Preliminaries}

Let $\{e_1,e_2,\ldots,e_n\}$ be an orthonormal basis of $\R^n$. We consider the $2^n$-dimensional real Clifford algebra $\R_{0,n}$ generated by $e_1,e_2,\ldots,e_n$ subject to the condition 
$$
x^2=-|x|^2 \mbox{ for all } x\in\mathbb{R}^n.
$$
where $|x|^2=x_1^2+\cdots +x_n^2$. 

This means that the basis elements satisfy the multiplication rules $e_ie_j=-e_je_i$ for $i\neq j$, . The elements $e_A : A\subseteq \N_n:=\{1,2,\ldots, n+1\}$ define a basis of $\R_{0,n}$, where $e_A=e_{h_1}\cdots e_{h_k}$ if $A=\{h_1, \ldots,h_k\}$ $(1\leq h_1<\cdots<h_k\leq n)$ and $e_\emptyset=e_0=1$. Therefore, any element $a\in\R_{0,n}$ may thus be written as $a=\sum_{A\inci\N_n}a_Ae_A$ where $a_A\in\R$ or still as $a=\sum_{k=0}^n[a]_k$, where {$[a]_k=\sum_{|A|=k}a_A e_A$} is a so-called $k$-vector part $(k\in\N_n^0:=\N_n\cup\{0\})$. Denoting the space of $k$-vectors by $\R_{0,n}^{(k)}$ we have $\R_{0,n}=\oplus_{k=0}^n\R_{0,n}^{(k)}$. This also introduces a grading into our algebra.

Hereby, the spaces $\R$ and $\R^n$ will be identified with 
 $\R_{0,n}^{(0)}$ and $\R_{0,n}^{(1)}$ respectively. This means that each element $x=(x_0, x_1,\ldots, x_n)\in\R^{n+1}$ can be written as
	\[x=x_0+ \sum_{i=1}^nx_ie_i\in\R_{1,n}^{(0)}\oplus\R_{1,n}^{(1)}
\]
and are often called {\it paravectors}. 


We will use the {\it conjugation}, an anti-automorphism which is defined by $\bar{a}=\sum_{A\inci\N_n}a_A\bar{e}_A$, where 
	\[\bar{e}_A:=(-1)^ke_{h_k}\cdots e_{h_1}=(-1)^{\frac{k(k+1)}{2}}e_A, \hspace{.5cm}\mbox{ if  } \hspace{.3cm} e_A=e_{h_1}\cdots e_{h_k}.
\]

For each $x\in \R_{0,n}^{(0)}\oplus\R_{0,n}^{(1)}$ we have 
\begin{equation}\label{Par Norm}
x\bar{x}=\bar{x}x=x_0^2+x_\epsilon^2 + x_1^2+\cdots+x_n^2=|x|^2.
\end{equation}
The extension of (\ref{Par Norm}) to a norm of $a\in\R_{0,n}$ is straightforward and leads to 
	\[|a|^2=[a\bar{a}]_0=[\bar{a}a]_0=\sum_A a_A^2.
\]

The following properties of the norm and conjugation in Clifford algebras are well-known and can be found in many sources, see for instance \cite{DSS}. 
\begin{proposition}
Let $a,b\in\Cl$, then
\begin{itemize}
  \item[(i)] $\ba{ab}=\bar{b}\; \bar{a}$,
	\item[(ii)] $|\bar{a}|=|-a|=|a|$ 
	\item[(iii)] $\left[a\bar{b}\right]_0=\big[\bar{a}b\big]_0=\langle a, b\rangle_{\R^{2^n}}$, where $\langle \cdot, \cdot\rangle_{\R^m}$ denote the standard scalar product in $\R^{m}$,
	\item[(iv)] $|ab|\leq 2^{n/2}|a||b|$,
	\item[(v)] if $b$ is such that $b\bar{b}=|b|^2$, $b\neq 0$, then $b$ is invertible and $|ab|=|ba|=|a||b|$.
\end{itemize}
\end{proposition}

Suppose $\Omega\subset\mathbb{R}^{n}$ is a bounded domain, with sufficiently smooth boundary $\Gamma$. We consider a domain $\Omega_T=[0,T]\times\Omega$. We will be interested in functions $f:\Omega_T \rightarrow\R_{0,n}$, which might be written as $f(x)=\sum_{A}f_A(x)e_A$ with $f_A$ being $\R$-valued. Property, such continuity, differentiability, integrability, and so on, can be ascribed coordinate-wise or directly. A left (unitary) module over $\R_{0,n}$ is a vector space $V$ together with an algebra morphism $L:\R_{0,n}\mapsto \mathrm{End}(V)$, or to say it more explicitly, there exists a linear transformation (also called left multiplication) $L(a)$ of $V$ such that
$$
L(ab+c)=L(a)L(b)+L(c)
$$ 
for all $a\in\R_{0,n}$, and $L(1)$ is the identity operator. In the same way we have a right (unitary) module if there is a so-called right multiplication $R(a)\in  \mathrm{End}(V)$ such that
$$
R(ab+c)=R(b)R(a)+R(c).
$$
Given either a left or a right multiplication we can always construct a right or a left multiplication by using any anti-automorphism of the algebra, for instance
$$
R(a)=L(\ba{a}).
$$
A bi-module is a module which is both a left- and a right-module, or with other words, a module where left and right multiplication commute, i.e. 
$$
L(a)R(b)=R(b)L(a), \mbox{ for all }a,b\in \R_{0,n}.
$$
If $V$ is a vector space of $\R_{0,n}$-valued function we consider the left (right) multiplication defined by pointwise multiplication
$$
(L(a)f)(x)=a(f(x)) \mbox{ and } (R(a)f)(x)=a(f(x)).
$$
Also a mapping $K$ between two right modules $V$ and $W$ is called a $\R_{0,n}$-linear mapping if
$$
K(fa+g)=K(f)a+K(g).
$$
This considerations give rise to the following right modules of $\R_{0,n}$-valued functions defined over any suitable subset $E$ inside the null-cone of $\R^{1,n}$:
\begin{itemize}
	\item $C^k(E,\Cl), k\in\mathcal{N}\cup\{0\}$-- the right module of all $\Cl$-valued functions, $k$-times continuously differentiable in $E$. It becomes a right Banach module with the norm
	$$
	\|f\|_{C^k}=\sup_{x\in E}\sum_{|\alpha|\leq k}|D^\alpha_w f(x)|
	$$
where $\alpha$ denotes a multi-index and $D^\alpha$ the corresponding partial derivative. In particular, we also have $C^\infty(E,\Cl):=\bigcap_{k=0}^\infty C^k(E,\Cl)$. Let us remark that the above norm is equivalent to the norm coming from the inductive limit.

\item By using the corresponding H\"older-semi-norm we can introduce $C^{0,\mu}(E, \Cl)$, $\mu\in(0,1]$, as the right Banach module of all $\mu$-H\"older continuous and $\Cl$-valued functions in $E$. 
\item $L_p(E,\Cl)$, $1\leq p<\infty$, denotes the right module of all equivalence classes of Lebesgue measurable functions $f:E\mapsto\Cl$ for which $|f|^p$ is integrable over $E$.  With the norm
  \[||f||_{L_p(E,\Cl)}:=\left(\int_E |f(\xi)|^p\;d\xi\right)^{1/p}<\infty,
\]
$L_p(E,\Cl)$ becomes a right Banach module.
\item $C^\infty_0(E,\Cl)$-- the space of all infinitely differentiable functions with compact support in $E$. An important property of the space $C^\infty_0(E,\Cl)$ is its density in the spaces $C^0(E,\Cl)$ and $L_p(E,\Cl)$.
\end{itemize}

Furthermore, the above proposition allows us to introduce the following sesquilinear form (also called symmetric inner product in the literature) for two functions $f,g:\Omega\rightarrow$ $\Cl$
$$
(f,g):=\ba{f}g.
$$
In particular, this sesquilinear form is a real-bilinear mapping and a Clifford-linear mapping in the second argument. 
Furthermore, this sesquilinear form gives rise to a real-valued positive definite inner product 
$$
\langle f,g\rangle=[(f,g)]_0
$$
with $\langle f,f\rangle=|f|^2$. This allows us to consider the space $L_2(E, \Cl)$ as a Hilbert module which is complete under the norm coming from the real-valued inner product
\begin{equation}\label{Pro Int}
\left\langle f,g\right\rangle_{2,R}=\int_E [\ba{f(\xi)} g(\xi)\,]_0 d\xi. 
\end{equation}
Since the real-valued inner product and its norm are coming from the restriction of a $\Cl$-valued sesquilinear form
$$
\left\langle f,g\right\rangle_{2}=\int_E \ba{f(\xi)} g(\xi)\, d\xi
$$
many proofs which only involve identities can be shown for the sesquilinear form and then reduced to the real-valued inner product on both sides. Because of this fact one usually considers the Hilbert module equipped with the $\Cl$-valued inner product $\langle\cdot,,\cdot\rangle_{2}$. In particular, a generalization of Riesz' representation theorem is valid in the sense that a linear functional $\phi$ is continuous if and only if it can be represented by an element $f_\phi\in V$ such that
$$
\phi(g)=\langle f_\phi,g\rangle_{2}.
$$
Additionally, there are some important inequalities involving the sesquilinear form and the norm coming from the real-valued inner product:
\begin{itemize}
\item Cauchy-Schwarz inequality: $|\langle f,g\rangle_2|\leq 2^{n/2}\|f\|_{L_p(E,\Cl)} \|g\|_{L_q(E,\Cl)}$ with $\frac{1}{p}+\frac{1}{q}=1$
\item $\|a f\|_{L_2(E,\Cl)}\leq 2^{n/2}|a|\|f\|_{L_2(E,\Cl)}$ for all $a\in\Cl$, but $\|a f\|_{L_2(E,\Cl)}=|a|\|f\|_{L_2(E,\Cl)}$ whenever $a$ is a paravector or belongs to the paravector group.
\item $\|f\|_{L_2(E,\Cl)}\leq |\langle f,f \rangle_2|\leq 2^{n/2}\|f\|_{L_2(E,\Cl)}$
\item $\|f\|_{L_2(E,\Cl)}\leq \sup_{\|g\|_{L_2(E,\Cl)} \leq 1}|\langle f,g \rangle_2|\leq 2^{n/2}\|f\|_{L_2(E,\Cl)}$
\end{itemize}
The same statements are also true for any other Clifford Hilbert module coming from a tensor product between the elements of the Clifford basis and elements of a real or complex Hilbert space~\cite{Held}. {\color{red} ADD REFFERENCE ``Held"}.


In the same way we can introduce $\mathcal{S}$ as the corresponding Schwartz space of rapidly decaying functions. Its dual space $\mathcal{S}^\prime$ is given by the continuous linear functionals is the space of tempered distributions.  Again, this can be either defined component-wise or via the sesquilinear form 
$\langle f,g \rangle_{2}$ but the space $\mathcal{S}^\prime$ is again considered as a right module. Let us remark that strictly speaking if we consider $\mathcal{S}$ as a Fr\'echet right module its algebraic dual $\mathcal{S}^\prime$ is the space of all left-$\Cl$-linear functionals over $\mathcal{S}$, but it can be identified with elements of a right-linear module by means of the standard anti-automorphism in the above mentioned way.
 
This allows us to introduce the {\it Sobolev} space $W_p^k(E,\Cl), k\in \mathbb{N}\cup\{0\},1\leq p<\infty$, as the right module of all functionals $f\in \mathcal{S}^\prime$ whose derivatives\footnote{$D^\alpha_w f=\frac{\partial^{|\alpha |}f}{\partial x_0^{\alpha_0}\cdots\partial x_n^{\alpha_n}}$ where $\alpha=(\alpha_0,\ldots,\alpha_n)\in\left(\N\cup\{0\}\right)^{n+1}$ is a multi-index and $|\alpha |=\alpha_0+\cdots+\alpha_n$.}	 $D^\alpha_w f$ for $|\alpha | \leq k$ belong to $L_p(E,\Cl)$, with norm
  $$
  \|f\|_{W_p(E,\Cl)}:=\left(\sum_{\|\alpha\| \leq k}\| D^\alpha_w f \|_{L_p(E,\Cl)}^p\right)^{1/p}.
  $$ 
 Alternatively, we also can consider the Sobolev space  $H_p^S(E,\Cl), s\in \mathbb{R},1\leq p<\infty$, as the right module of all functionals $f\in \mathcal{S}^\prime$ whose extension to $\mathbb{R}^{n+1}$ satisfies
 $$
\hat{f} ( \cdot ) < \cdot >^s \in L_p(E,\Cl)
 $$ 
where $\hat f$ denotes the Fourier transform and $<\xi>=(1+|\xi |^2)^{1/2}$ denotes the Japanese bracket. We remark that when $E$ has the extension property we have $W_p^k(E,\Cl)=H_p^k(E,\Cl)$.
% 
 
For more details we refer to the classic book~\cite{BDS}.

%%%%%%%% Dirac operator in our setting

On the set $C^1(\Om,\Cl)$ we define respectively the classic left and the right Dirac operator by: 
\begin{equation}\label{Op Dif}
D[f]:=\sum_{i=1}^n e_i\frac{\partial f}{\partial x_i}, \hspace{.5cm} [f]D:=\sum_{i=1}^n\frac{\partial f}{\partial x_i}e_i.
\end{equation}
For simplicity of notation we continue to write $\partial_{i}$ instead of $\displaystyle\frac{\partial}{\partial x_i}$. Functions belonging to the kernel of this operator are called left- or right-monogenic functions. In case of left-monogenic functions one simply speaks also of monogenic functions. 

Let $\Delta_{n}$ be the $n$-dimensional Laplace operator. It is easy to prove that the equalities
\begin{equation}\label{Fac Lap}
D^2=-\Delta_{n},
\end{equation}
hold.

In this paper we will consider the non-constant Dirac operator with possibly distributional coefficients of the type
$$
\mathcal{D}f=\frac{\partial}{\partial x_0}+a(x_0)Df :=\frac{\partial}{\partial x_0}+a(x_0)\sum_{i=1}^n e_i\frac{\partial f}{\partial x_i}.
$$

\section{CK-extension in the case of Dirac operators with non-constant coefficients}

The classic CK-extension method allows one to construct monogenic functions over domains in $\R^{n+1}$ from analytic functions over $\R^n$. 

\begin{theorem}
    Let $\un{\Omega}\in \R^n$ be an open domain, $u_0 \in C^\infty(\un{\Omega})$ and $a$ be an integrable function in $[0,T]$, then the CK-extension
    \[
        CK[u_0](x_0, \un x) = \sum_{k = 0}^\infty \frac{(-1)^k}{k!}A^k(x_0)\un{\D}^ku_0(\un{x})
    \]
    solves the following Cauchy problem
    \[
    \begin{cases}
    \partial_{x_0} u(x_0,\un{x}) + a(x_0)\un{\D} u(x_0,\un{x}) = 0, & \text{with } x_0 \in (0, T] \text{ and } \un{x} \in \un{\Omega}, \\
    u(0,\un{x}) = u_0(\un{x}),\;\; \un{x}\in \un{\Omega},
    \end{cases}    \]
     where $A(x) = \int_0^x a(t) dt.$ 

    
\end{theorem}
\begin{proof}
We look into the %It is reasonable to expect that the 
appropriate CK-extension for the Cauchy problem (\ref{CP}) %will be 
of the form
\[
CK[u_0](x_0, \un{x})=\sum_{k = 0}^\infty \frac{(-1)^k}{k!}\phi(x_0)^k\un{\D}^ku_0(\un{x}),
\]
for some non-zero differentiable function $\phi$ in $[0,T]$.

As $CK[u_0]$ should be a solution to the differential equation

\[
(\partial_0 + a(x_0)\un{\D})CK[u_0] =0,
\]
then
\begin{align*}
    0 
    &= (\partial_0 + a(x_0)\un{\D})\Big(\sum_{k = 0}^\infty \frac{(-1)^k}{k!}\phi(x_0)^k\un{\D}^ku_0(\un{x}) \Big) \\
   & = \partial_0[u_0(\un{x}) + \sum_{k = 1}^\infty \frac{(-1)^k}{k!}\phi(x_0)^k\un{\D}^ku_0(\un{x})]\\
    &+a(x_0)\sum_{k = 0}^\infty \frac{(-1)^k}{k!}\phi(x_0)^k\un{\D}^{k+1}u_0(\un{x}) \\
      & = \sum_{k = 1}^\infty \frac{(-1)^k}{(k-1)!}\phi(x_0)^{k-1}\phi'(x_0)\un{\D}^ku_0(\un{x}) \\
    &+\sum_{k = 0}^\infty \frac{(-1)^k}{k!}a(x_0)\phi(x_0)^k\un{\D}^{k+1}u_0(\un{x}) \\
    &= \sum_{k = 0}^\infty \frac{(-1)^{k+1}}{k!}\phi(x_0)^k\phi'(x_0)\un{\D}^{k+1}u_0(\un{x}) \\
    &+\sum_{k = 0}^\infty \frac{(-1)^k}{k!}a(x_0)\phi(x_0)^k\un{\D}^{k+1}u_0(\un{x}) \\
    &= \sum_{k = 0}^\infty [-\phi'(x_0) + a(x_0)]\frac{(-1)^k}{k!}\phi(x_0)^k\un{\D}^{k+1}u_0(\un{x}).
\end{align*}
Since this equality has to hold for every $u_0$ and $\phi$ is non-zero then,
\[  
-\phi'(x) + a(x) = 0. 
\]
This is an ODE with a classical solution
\[
\phi(x) = \int_0^x a(t) dt + C, \qquad C\in \mathbb{R}
\]
For shorthand, let us define $A(x) = \int_0^x a(t) dt$.

As $CK[u_0]$ should also be a solution to the boundary condition of the Cauchy problem (\ref{CP}), then
%Given the boundary condition of the Cauchy problem (\ref{CP}), we get
\begin{align*}
    u_0(\un x) 
    &= CK[u_0](0, \un x)\\
    &=\sum_{k = 0}^\infty \frac{(-1)^k}{k!}\phi(0)^k\un{\D}^ku_0(\un{x})\\
    &= C^0u_0(\un{x})+ \sum_{k = 1}^\infty \frac{(-1)^k}{k!}C^k\un{\D}^ku_0(\un{x}).
\end{align*}
This leads to the condition %Since this condition has to hold for every suitable $u_0$,
\[
C = 0.
\]
As such, when $a$ is integrable, %ESCOLHER O QUE QUERES USAR AQUI
\[
CK[u_0](x_0, \un{x}) = \sum_{k = 0}^\infty \frac{(-1)^k}{k!}A^k(x_0)\un{\D}^ku_0(\un{x})
\]
is an explicit solution to the Cauchy problem (\ref{CP}). Aditionally, due to the properties of the CK-extension, when $a$ is infinitely diferentiable the solution is also unique. 
\end{proof}
For infintely differentiable $u_0$, there is a simple norm estimate for the CK-extension obtained above.

Given that  NO PREL PÔR ESTAS NORMAS
\[
\hnorm{(x_0\un{\D})u_0}{p} \leq \norm{x_0}\hnorm{u_0}{\infty}. 
\]
It is also true that
\[
\hnorm{(x_0\un{\D})^ku_0}{p} \leq \norm{x_0}\hnorm{(x_0\un{\D})^{k-1}u_0}{\infty}. 
\]
Repeating this step $k$ times:
\[
\hnorm{(x_0\un{\D})^ku_0}{p} \leq \norm{x_0}^k \hnorm{u_0}{\infty}. 
\]
Adapting to our case we have, with $\sup_{t \in [0, T]} (\norm{A(t)}) = A$,

\[
\|(A(\cdot)\un{\D} )^k u_0\|_{H^p([0,T]\times \un \Omega)}  \leq A^k \hnorm{u_0}{\infty}
\]
As such, a possible energy estimate for the CK-extension is 
\begin{align*}
\| CK[u_0]\|_{H^p([0,T]\times \un \Omega)}    &= \| \sum_{k = 0}^\infty \frac{(-1)^k}{k!}A(\cdot)^k\un{\D}^ku_0\|_{H^p([0,T]\times \un \Omega)} \\
        &\leq \sum_{k = 0}^\infty \frac{1}{k!}\|A(\cdot)^k\un{\D}^ku_0\|_{H^p([0,T]\times \un \Omega)} \\
        &\leq  \left( \sum_{k = 0}^\infty \frac{A^k}{k!} \right) \hnorm{u_0}{\infty}\\
        &= e^{A}\hnorm{u_0}{\infty}.
\end{align*}
%%%%%%%%%%%%%%%%%%%%%%%%%%%%%%%%%%%%%%%%%%%%%
%%%%%%%%%%%%%%%%%%%%%%%%%%%%%%%%%%%%%%%%%%%%%
\section{Energy estimates for Dirac operators with non-constant coefficients}

Let us now take a closer look into energy estimates using a different approach. Considering the case $\Omega=\R^n$ and applying the Fourier transform in the variable $\un{x}$ to~(\ref{CP}) we get
\begin{equation}
\begin{cases}
\partial_{x_0}\hat{u}(x_0,\xi) = i\xi a(x_0)\hat{u}(x_0,\xi), \qquad \xi\in\R^n\\
\hat{u}(0,\xi) = \hat{u}_0(\xi).
\end{cases}
\label{eq3.4-td}
\end{equation}
where $\hat{u}$ denotes the Fourier transform in $\R^n$. Let us furthermore consider $a \in \text{Lip}([0, T]), \, a(x_0) \geq a_0 > 0$. Additionally, we will denote $\hat{u}(x_0,\cdot)=\hat{u}(x_0)$.

Here we define the energy as
\[
E(x_0) := \mathrm{Re}\left(\langle a(x_0)\hat{u}(x_0), \hat{u}(x_0) \rangle\right),
\]
and we need to estimate its variations in the variable $x_0$. Using the fact that $a \in \text{Lip}([0, T])$ and $a(x_0) \geq a_0 > 0$ we have immediately
\begin{equation}
\min_{x_0 \in [0,T]} \{a(x_0)\}\|\hat{u}\|^2\leq E(x_0) \leq \max_{t \in [0,T]} \{a(x_0)\} \|\hat{u}\|^2 .
\label{eq3.5-td}
\end{equation}

In particular, in this case the condition on $a$ ensures the existence of two strictly positive constants $c_0 $ and $ c_1 $ such that $ c_0 = \min_{t \in [0,T]} a(x_0) $ and $ c_1 = \max_{t \in [0,T]} a(x_0) $.

Therefore, inequality~(\ref{eq3.5-td}) can be written as
\begin{equation}
c_0 \|\hat{u}(x_0)\|^2 \leq E(x_0) \leq c_1 \|\hat{u}(x_0)\|^2.
\label{eq3.6-td}
\end{equation}

A straightforward calculation, together with~(\ref{eq3.6-td}), provides us with the following estimate:
\begin{eqnarray*}
\partial_{x_0}E(x_0) & = & \mathrm{Re}\left(\langle \partial_{x_0}a(x_0)\hat{u}(x_0), \hat{u}(x_0) \rangle + \langle a(x_0)\partial_{x_0}\hat{u}(x_0), \hat{u}(x_0)\right. \rangle\\
&&  \left. + \langle a(x_0)\hat{u}(x_0), \partial_{x_0}\hat{u}(x_0) \rangle \right)\\
& = & \mathrm{Re}\left(\langle \partial_{x_0}(x_0)\hat{u}(x_0), \hat{u}(x_0) \rangle \right)\\
&& +  \mathrm{Re}\left(i\xi\langle a(x_0)^2\hat{u}(x_0), \hat{u}(x_0) \rangle - \langle a(x_0)\hat{u}(x_0), a(x_0)\hat{u}(x_0) \rangle i\overline{\xi}\right).
\end{eqnarray*}
With $w=\langle a(x_0)^2\hat{u}(x_0), \hat{u}(x_0) \rangle$ and $z=i\xi$ the last two terms are of the form $zw-\overline{w}\;\overline{z}=zw-\overline{zw}$ so that the real part is zero. Therefore, we have

\begin{equation}
\partial_{x_0}E(x_0)=\mathrm{Re}\left(\langle \partial_{x_0}a(x_0)\hat{u}(x_0), \hat{u}(x_0) \rangle \right)\leq \sup_{x_0 \in [0,T]} |\partial_{x_0}a(x_0)| \|\hat{u}(x_0)\|^2.
\label{eq3.7-td}
\end{equation}
Putting $c' := c_0^{-1} \sup_{x_0 \in [0,T]} |\partial_{x_0}a(x_0)|$, we get from~(\ref{eq3.7-td}) using~(\ref{eq3.6-td}) that
\begin{equation}
\partial_{x_0}E(x_0)  \leq c'E(x_0).
\label{eq3.8-td}
\end{equation}
The last estimate allows us to apply Gronwall's lemma to~(\ref{eq3.8-td}), which leads to the existence of a constant $c > 0$ independent of $x_0 \in [0, T]$ such that
\begin{equation}
E(x_0) \leq cE(0).\label{eq3.9-td}
\end{equation}
Therefore, putting together~(\ref{eq3.9-td}) and~(\ref{eq3.6-td}), we obtain
\begin{equation*}
c_0\|\hat{u}(x_0)\|^2 \leq E(x_0) \leq cE(0) \leq cc_1\|\hat{u}(0)\|^2.
\end{equation*}
This means that there exists a constant $C > 0$ independent of $t$ such that $\|\hat{u}(x_0)\|^2 \leq C\|\hat{u}(0)\|^2$. From this we have
\begin{equation*}
\|u(x_0)\|^2  \leq C\|u_0\|^2,
\end{equation*}
as required.

\section{Notion of Very Weak Solution}

In this section, we revisit the concept of \textbf{very weak solutions}. Since this notion depends intrinsically on the structure of the differential equation under consideration, we adapt it to our specific framework involving distributions:
\begin{itemize}
    \item $a \in \mathcal{D}'([0, T])$ (for the coefficient),
    \item $f \in \mathcal{D}'([0, T]) \otimes \overline{H^{-\infty}(\mathbb{R})}$ (for the source term).
\end{itemize}

To define the solution rigorously, we employ a regularization procedure for both the distributional coefficient $a$ and the source term $f$. This is accomplished via convolution with a suitable mollifier $\psi$, which produces families of smooth functions $(a_\varepsilon)_\varepsilon$ and $(f_\varepsilon)_\varepsilon$ (regularized coefficients and regularized source terms), which
\begin{equation}
\label{eq1.7}
a_\varepsilon = a * \psi_{\omega(\varepsilon)} \quad \text{and} \quad f_\varepsilon = f(\cdot) * \psi_{\omega(\varepsilon)},
\end{equation}
with
\[
\psi_{\omega(\varepsilon)}(x_0) = (\omega(\varepsilon))^{-1} \psi(x_0 / \omega(\varepsilon)),
\]
where $\omega(\varepsilon) > 0$ satisfies  $\omega(\varepsilon) \to 0$ as $\varepsilon \to 0$, and $\psi$ is a Friedrichs-mollifier, which means that,
\[
\psi \in C_0^\infty(\mathbb{R}), \quad \psi \geq 0, \quad \int \psi = 1.
\]

Let us formally define nets of functions as follows.

\begin{definition}
\label{def-1.3-hip}
\begin{enumerate}
    \item[(i)] A net of functions $(f_\varepsilon)_\varepsilon \in C^\infty(\mathbb{R})^{(0,1]}$ is said to be $C^\infty$-moderate if for all $K \Subset \mathbb{R}$ and for all $\alpha \in \mathbb{N}_0$, there exist $N \in \mathbb{N}_0$ and $c > 0$ such that
    \[
    \sup_{t \in K} \| \partial^\alpha f_\varepsilon(x_0) \| \leq c \varepsilon^{-N-\alpha},
    \]
    for all $\varepsilon \in (0, 1]$, where $K \Subset \mathbb{R}$ means that $K$ is a compact set in $\mathbb{R}$.
    
    \item[(ii)] A net of functions $(u_\varepsilon)_\varepsilon \in C^\infty([0, T]; H^s)^{(0,1]}$ is $C^\infty([0, T]; H^s)$-moderate if there exist $N \in \mathbb{N}_0$ and $c_k > 0$ for all $k \in \mathbb{N}_0$ such that
    \[
    \| \partial_{x_0}^k u_\varepsilon(x_0, \cdot) \|_{H_R^s} \leq c_k \varepsilon^{-N-k},
    \]
    for all $x_0 \in [0, T]$ and $\varepsilon \in (0, 1]$.
    
    \item[(iii)] We say that a net of functions $(u_\varepsilon)_\varepsilon \in C^\infty([0, T]; H^{-\infty}(s))^{(0,1]}$ is $C^\infty([0, T]; H^{-\infty}(s))$-moderate if there exists $\eta > 0$, and for all $p \in \mathbb{N}_0$, there exist $c_p > 0$ and $N_p > 0$ such that
    \[
    \| e^{-\eta R^{\frac{1}{2s}}} \partial_{x_0}^p u_\varepsilon(x_0, \cdot) \|_{L^2(\Omega)} \leq c_p \varepsilon^{-N_p-p},
    \]
    for all $x_0 \in [0, T]$ and $\varepsilon \in (0, 1]$.
\end{enumerate}
\end{definition}
When $\omega(\varepsilon) = \varepsilon$, then the net $(a_\varepsilon)_\varepsilon$ from~(\ref{eq1.7}) satisfies $C^\infty$-moderateness. This is expected, since distribution regularizations are moderate  per se as the structure theorems of distribution theory which
\begin{equation}
\mathcal{E}'(\mathbb{R}) \subset \{ C^\infty\text{-moderate families} \}. \label{eq1.8}
\end{equation}

Thus, as seen in~(\ref{eq1.8}), we find that although the distributions in $\mathcal{E}'(\mathbb{R})$ do not provide solutions to our Cauchy problem, solutions might still be available in other settings.

The moderateness hypothesis provides the necessary framework for solution recovery via~(\ref{eq:1.6}) in solvable cases. We observe that classical Friedrichs mollification methods exhibit limitations in this context, thereby justifying the implementation of our $\omega(\varepsilon)$-based regularization approach.

Now let us introduce a notion of a ``very weak solution'' for the Cauchy problem:
\[
\begin{cases}
\partial_{x_0} u(x_0,\un{x}) + a(x_0) \mathcal{D} u(x_0,\un{x})= f(x_0,\un{x}), & (x_0,\un{x}) \in [0, T]\times \Omega, \\
u(0,\un{x}) = u_0{(\un{x})}, \qquad \un{x}\in \Omega.
\end{cases}
\]

\vspace*{0.1cm}
\begin{definition}
\label{def-1.4-hip}
Let $s$ be a real number.
\begin{enumerate}
    \item[(i)] We say that the net $(u_\varepsilon)_\varepsilon \subset C^\infty([0, T]; H_R^s)$ is a very weak solution of $H^s$-type of the Cauchy problem (1.9) if there exist $C^\infty$-moderate regularization $a_\varepsilon$ of the coefficient $a$, $C^\infty([0, T]; H_R^s)$-moderate regularization $f_\varepsilon$ of $f$, such that $(u_\varepsilon)_\varepsilon$ solves the following regularized problem:
    \[
    \begin{cases}
    \partial_t u_\varepsilon(x_0) + a_\varepsilon(x_0) \mathcal{D} u_\varepsilon(x_0) = f_\varepsilon(x_0), & t \in [0, T], \\
    u_\varepsilon(0) = u_0 \in L^2(\mathbb{R}^n).
    \end{cases} 
    \]
    for all $\varepsilon \in (0, 1]$, and is $C^\infty([0, T]; H_R^s)$-moderate.
    
    \item[(ii)] The net $(u_\varepsilon)_\varepsilon \subset C^\infty([0, T]; H^{-\infty}(s))$ is a very weak solution of $H^{-\infty}(s)$-type of the Cauchy problem (1.9) if there exist $C^\infty$-moderate regularization $a_\varepsilon$ of the coefficient $a$, $C^\infty([0, T]; H^{-\infty}(s))$-moderate regularization $f_\varepsilon(x_0)$ of $f(x_0)$, such that $(u_\varepsilon)_\varepsilon$ solves the regularized problem (1.10) for all $\varepsilon \in (0, 1]$, and is $C^\infty([0, T]; H^{-\infty}(s))$-moderate.
\end{enumerate}
\end{definition}

\section{Very weak solutions for Dirac operators with distributional coefficients}

\begin{theorem}{(\bf Theorem 1.5)}(Existence)
\label{th1.5}\\
Let $T > 0$ and $s \in \mathbb{R}$.
\begin{enumerate}
    \item[(i)] Let $a = a(x_0)$ be a positive distribution with compact support included in $[0, T]$, such that $a \geq a_0$ for some constant $a_0 > 0$. Let $u_0 \in H_R^s$ and $f \in D'([0, T]) \otimes \overline{H}_R^s$. Then the Cauchy problem (\ref{CP}) has a very weak solution of $H^s$-type.
    
    \item[(ii)] Let $a = a(x_0)$ be a nonnegative distribution with compact support included in $[0, T]$, such that $a \geq 0$. Let $u_0\in H^{-\infty}(s)$ and $f \in D'([0, T]) \otimes \overline{H^{-\infty}(s)}$. Then the Cauchy problem (\ref{CP}) has a very weak solution of $H^{-\infty}(s)$-type.
\end{enumerate}
\end{theorem}

Now we show that the very weak solution of the Cauchy problem~(\ref{CP}) is unique in an appropriate sense.

\begin{definition}{\bf Definition 1.6.}
\label{def1.6}
The net $(u_\varepsilon)_\varepsilon$ is $C^\infty$-negligible if for all $K \Subset \mathbb{R}$, for all $\alpha \in \mathbb{N}$ and for all $\ell \in \mathbb{N}$ there exists a positive constant $c$ such that
\[
\sup_{x_0 \in K} \| \partial^\alpha u_\varepsilon(x_0) \| \leq c \varepsilon^\ell,
\]
for all $\varepsilon \in (0, 1]$.\\
\end{definition}
In the present framework, it suffices to consider $K = [0, T]$, as all temporal distributions of interest are supported in $[0, T]$.

\subsection*{Proof of Theorem 1.5}

\noindent
\begin{proof}[Proof of Theorem 1.5.]
(i) Suppose the coefficient $a = a(x_0)$ is a distribution with compact support in $[0, T]$. In this case, the classical formulation of problem~(\ref{CP}) may fail in the distributional framework due to the well-known difficulties in defining products of distributions. To address this challenge, we instead consider a regularized version of~\eqref{CP}. Specifically, through mollification of $a$ using $\psi \in C_{0}^{\infty}(\mathbb{R})$, we generate a family of smooth coefficients $(a_{\varepsilon})_{\varepsilon}$, which enables a rigorous analysis.

For this, we take $\psi\in C_{0}^{\infty}(\mathbb{R})$, $\psi\geq 0$ with $\int\psi=1$, and $\omega(\varepsilon)>0$ such that $\omega(\varepsilon)\to 0$ as $\varepsilon\to 0$ to be chosen later. Then, we define $\psi_{\omega_{\varepsilon}}$ and $a_{\varepsilon}$ by

\[
\psi_{\omega_{\varepsilon}}(x_0):=\frac{1}{\omega(\varepsilon)}\psi\left(\frac{x_0}{\omega(\varepsilon)}\right)
\]

and

\[
a_{\varepsilon}(x_0):=(a*\psi_{\omega(\varepsilon)})(x_0)
\]

for all $x_0\in[0,T]$, respectively. Using these representations of $\psi_{\omega_{\varepsilon}}$ and $a_{\varepsilon}$ and identifying the measure $a(x_0)$ with its density, we get

\[
a_{\varepsilon}(x_0)=(a*\psi_{\omega(\varepsilon)})(x_0)=\int\limits_{\mathbb{R}}a(x_0-\tau)\psi_{\omega(\varepsilon)}(\tau)d\tau=\int\limits_{\mathbb{R}}a(x_0-\omega(\varepsilon)\tau)\psi(\tau)d\tau
\]

\[
=\int\limits_{K}a(x_0-\omega(\varepsilon)\tau)\psi(\tau)d\tau\geq a_{0}\int\limits_{K}\psi(\tau)d\tau:=\widetilde{a_{0}}>0,
\]

where we have used that $a(x_0)$ is a positive distribution with compact support (hence a Radon measure) and $\psi\in C_{0}^{\infty}(\mathbb{R})$, supp $\psi\subset K$, $\psi\geq 0$ in above. Here, note that $\widetilde{a_{0}}$ does not depend on $\varepsilon$.

We also note that by virtue of the structure theorem for compactly supported distributions, there exist a natural number $L$ and positive constant $c$ such that for all $k\in\mathbb{N}_{0}$ and $x_0\in[0,T]$ we have

\begin{equation}
   |\partial_{x_0}^{k}a_{\varepsilon}(x_0)|\leq c(\omega(\varepsilon))^{-L-k}.
\label{eq3.3-hipoel} 
\end{equation}


Thus, $a_{\varepsilon}$ is $C^{\infty}$-moderate regularisation of the coefficient $a(x_0)$ under appropriate conditions on $\omega(\varepsilon)$, then fixing $\varepsilon\in(0,1]$ we consider the following regularised problem

\begin{equation}
\left\{
\begin{array}{l}
\partial_{x_0}u_{\varepsilon}(x_0)+a_{\varepsilon}(x_0)\mathcal{D}u_{\varepsilon}(x_0)=0,\ x_0\in[0,T], \\
u_{\varepsilon}(0)=u_{0}\in L^{2}(\mathbb{R}^n), 
\end{array}
\right. 
\label{eq3.4-hipoel} 
\end{equation}

where $u_{0}\in H^s$, $a_{\varepsilon}\in C^{\infty}[0,T]$. Then, Theorem~\ref{th1.1} implies that the regularised problem~(\ref{eq3.4-hipoel}) has a unique solution in the space $C([0,T]; H^s)$. Actually, noting \(a_{\varepsilon}\in C^{\infty}([0,T])\) and differentiating both sides of the equation~(\ref{eq3.4-hipoel}) in \(x_0\) inductively, one can see that this unique solution is from \(C^{\infty}([0,T];H{\varepsilon})\).\\

Thus, recalling $S(x_0):= a(x_0)$ 
by the proof of Theorem~\ref{th1.1} or Theorem~\ref{th1.2} with \(f\equiv 0\)), and noting~(\ref{eq3.3-hipoel}), we get

\[
\|\partial_{x_0}S(x_0)\|\leq C|\partial_{x_0}a_{\varepsilon}(x_0)|\leq C\omega(\varepsilon)^{-L-1}.
\]

Then, from~\ref{eq3.4-hipoel}  \(f\equiv 0\)), Gronwall's lemma with \(f\equiv 0\)) imply that

\begin{equation}
\|u_{\varepsilon}(x_0,\cdot)\|^{2}_{ H^s(\mathbb{R}^n)} \leq C\exp(c\omega(\varepsilon)^{-L-1}T)\|u_{0}\|^{2}_{ H^s(\mathbb{R}^n)}. ~\label{eq3.5-hipoel}
\end{equation}

If we take \((\omega(\varepsilon))^{-L-1}\approx\log\varepsilon\), then~(\ref{eq3.5-hipoel}) becomes

\[
\|u_{\varepsilon}(x_0,\cdot)\|^{2}_{ H^s(\mathbb{R}^n)} \leq C\varepsilon^{-L-1}\|u_{0}\|^{2}_{ H^s(\mathbb{R}^n)},
\]

with possibly new constant \(L\).

Now, to obtain that \(u_{\varepsilon}\) is \(C^{\infty}([0,T];H^{\varepsilon})\)-moderate, we need to show that for all \(x_0\in[0,T]\) and \(\varepsilon\in(0,1]\),

\begin{equation}
\|u_{\varepsilon}(x_0,\cdot)\|_{ H^s(\mathbb{R}^n)}\leq C\varepsilon^{-L}
\label{eq3.6-hipoel}
\end{equation}

hold for some \(L>0\). Indeed, once we prove this, then acting by the iterations of \(\partial_{t}\) on the equality

\[
\partial_{x_0}u_{\varepsilon}(x_0)=-a_{\varepsilon}(x_0){\mathcal{D}}u_{\varepsilon}(x_0),
\]

and taking it in \(L^{2}({\mathbb{R}^n})\)-norms, we conclude that \(u_{\varepsilon}\) is \(C^{\infty}([0,T];H^{\varepsilon})\)-moderate. 

Since \(u_{\varepsilon}\) is \(C^{\infty}([0,T];H^{s})\)-moderate, by the Definition~\ref{def-1.4-hip} we conclude that the Cauchy problem (\ref{CP}) has a very weak solution.\\

Now we prove Part (ii). Similarly as in Part (i), in this case one gets that for \(a_{\varepsilon}(x_0)\geq 0\) there are \(L\in\mathbb{N}\) and \(c_{1}>0\) such that

\begin{equation}
|\partial_{x_0}^{k}a_{\varepsilon}(x_0)|\leq c_{1}(\omega(\varepsilon))^{-L-k}, 
\label{eq3.7-hipoel}
\end{equation}

for all \(k\in\mathbb{N}_{0}\) and \(x_0\in[0,T]\), which means that \(a_{\varepsilon}(x_0)\) is a \(C^{\infty}\)-moderate regularisation of \(a(x_0)\). Then, fixing \(\varepsilon\in(0,1]\), we consider the following regularised problem

\begin{equation}
\left\{
\begin{array}{l}
\partial_{x_0}u_{\varepsilon}(x_0)+a_{\varepsilon}(x_0){\mathcal{D}}u_{\varepsilon}(x_0)=0,\ x_0\in[0,T], \\
u_{\varepsilon}(0)=u_{0}\in L^{2}(\mathbb{R}^n),
\end{array}
\right. 
\label{eq3.8-hipoel}
\end{equation}

where \(u_{0}\in H_{{(s)}}^{-\infty}\) and \(a_{\varepsilon}\in C^{\infty}[0,T]\). Then, we can usethe Theorem~\ref{th1.2}, which implies that the equation~(\ref{eq3.8-hipoel}) has a unique solution in the space \(u\in C^{1}([0,T];H_{{(s)}}^{-\infty})\) for any \(s\). Actually, this unique solution is from \(C^{\infty}([0,T];H_{{(s)}}^{-\infty})\), which can be checked by differentiating both sides of the equation~(\ref{eq3.8-hipoel}) in \(x_0\) inductively noting that \(a_{\varepsilon}\in C^{\infty}([0,T])\). Applying Theorem~\ref{th1.2} (iii) to the equation~(\ref{eq3.8-hipoel}), using the inequality

\[
|\partial_{x_0}a_{\varepsilon}(x_0)|\leq C(\omega(\varepsilon))^{-L-1}.
\]
This completes the proof of Theorem~\ref{th1.5}. 
\end{proof}
Now we prove Theorem~\ref{th1.7}.


%%%%%%%%

\begin{theorem}{(\bf Theorem 1.7)}
\label{th1.7}(Uniqueness)\\
Let $T > 0$.
\begin{enumerate}
    \item[(i)] Let $a = a(x_0)$ be a positive distribution with compact support included in $[0, T]$, such that $a(x_0) \geq a_0$ for some constant $a_0 > 0$. Let $u_0 \in H^s$ and $f \in L^2([0, T]; H^s)$ for some $s \in \mathbb{R}$. Then there exists an embedding of the coefficient $a(x_0)$ into $G([0, T])$, such that the Cauchy problem~(\ref{CP}) has a unique solution $u \in L^2([0, T]; H^s)$.
    
    \item[(ii)] Let $a = a(x_0) \geq 0$ be a nonnegative distribution with compact support included in $[0, T]$. Let $u_0\in H^{-\infty}(s)$ and $f \in L^2([0, T]; H^{-\infty}(s))$ for some $s \in \mathbb{R}$. Then there exists an embedding of the coefficient $a(x_0)$ into $L^2([0,T])$, such that the Cauchy problem~(\ref{CP}) has a unique solution $u \in L^2([0,T]; H_{(s)}^{-\infty})$.
\end{enumerate}
\end{theorem}
Now we give the consistency result, which means that very weak solutions recapture the classical solutions in the case the latter exist. 

Denote by $L_{1}^{\infty}([0,T])$ the space of bounded functions on $[0,T]$ with the derivative also in $L^{\infty}$.

\vspace{0.5cm}


\textbf{Proof of Theorem~\ref{th1.7}}\\
(i) We assume that the Cauchy problem has another solution $v \in L^2([0, T]; H^s)$. At the level of representatives this means

\[
\begin{cases}
\partial_{x_0} (u_\varepsilon - v_\varepsilon)(x_0) + a_\varepsilon(x_0)\mathcal{D}(u_\varepsilon - v_\varepsilon)(x_0) = \rho_\varepsilon(x_0), & x_0 \in [0, T], \\
(u_\varepsilon - v_\varepsilon)(0) = 0,
\end{cases}
\]

with  $\rho_\varepsilon = ( a_\varepsilon(x_0)-\widetilde{a}_\varepsilon(x_0) )\mathcal{D}v_\varepsilon(x_0)$, where $(\widetilde{a}_\varepsilon)_\varepsilon$ is an approximation corresponding to $u_\varepsilon$. Since $(a_\varepsilon)_\varepsilon \sim (\widetilde{a}_\varepsilon)_\varepsilon$, we have that $\rho_\varepsilon$ is $C^\infty([0, T]; H^s)$-negligible. Let us write this as in the following first order equation
\[
\partial_{x_0} w_{\varepsilon} +  a_\varepsilon(x_0)\mathcal{D}  w_{\varepsilon} =\rho_\varepsilon,
\]
where
\[
w_{\varepsilon} := u_\varepsilon - v_\varepsilon.
\]


Using the Fourier transform,
\[
\partial_{x_0} \hat{w}_\varepsilon(x_0, \xi) + i \xi a_\varepsilon(x_0, \xi) \hat{w}_\varepsilon(x_0, \xi) = \hat{\rho}_\varepsilon(x_0, \xi),
\]

for all $\xi \in \mathbb{R}^n$. We define the energy
\[
E_\varepsilon(x_0, \xi) := \mathrm{Re}(\langle a_\varepsilon(x_0, \xi) \hat{w}_\varepsilon(x_0, \xi), \hat{w}_\varepsilon(x_0, \xi)\rangle)
\]
Since $a_\varepsilon(x_0)$ is continuous, then from the definition of the energy we get
\begin{equation}
\label{eq3.12}
c_0 |\hat{w}_\varepsilon(x_0, \xi)|^2 \leq E_\varepsilon(x_0, \xi) \leq c_1 |\hat{w}_\varepsilon(x_0, \xi)|^2
\end{equation}
for some positive constants $c_0$ and $c_1$. Then, a direct calculation gives that
\begin{gather*}
\partial_{x_0} E_\varepsilon(x_0, \xi) = \mathrm{Re}\big(\langle \partial_{x_0} a_\varepsilon(x_0, \xi)\hat{w}_\varepsilon(x_0, \xi), \hat{w}_\varepsilon(x_0, \xi)\rangle + \langle a_\varepsilon(x_0, \xi) \partial_{x_0} \hat{w}_\varepsilon(x_0, \xi), \hat{w}_\varepsilon(x_0, \xi)\rangle\\ + \langle a_\varepsilon(x_0, \xi) \hat{w}_\varepsilon(x_0, \xi), \partial_{x_0} \hat{w}_\varepsilon(x_0, \xi)\rangle\big)\\             
= \mathrm{Re}\big(\langle\partial_{x_0} a_\varepsilon(x_0, \xi) \hat{w}_\varepsilon(x_0, \xi), \hat{w}_\varepsilon(x_0, \xi)\rangle + i \xi \langle a^2_\varepsilon(x_0, \xi)\hat{w}_\varepsilon(x_0, \xi), \hat{w}_\varepsilon(x_0, \xi)\rangle\\
- i \xi \langle a_\varepsilon(x_0, \xi)\hat{w}_\varepsilon(x_0, \xi), a_\varepsilon(x_0, \xi) \hat{w}_\varepsilon(x_0, \xi)\rangle + \langle a_\varepsilon(x_0, \xi)\hat{\rho}_\varepsilon(x_0, \xi), \hat{w}_\varepsilon(x_0, \xi)\rangle +\\ \langle a_\varepsilon(x_0, \xi)\hat{w}_\varepsilon(x_0, \xi), \hat{\rho}_\varepsilon(x_0, \xi)\rangle\big)\\
= \mathrm{Re}\big( \langle \partial_{x_0} a_\varepsilon(x_0, \xi)\hat{w}_\varepsilon(x_0, \xi), \hat{w}_\varepsilon(x_0, \xi) \rangle + \langle a_\varepsilon(x_0, \xi)\hat{\rho}_\varepsilon(x_0, \xi), \hat{w}_\varepsilon(x_0, \xi)\rangle \\+ \langle\hat{w}_\varepsilon(x_0, \xi), a_\varepsilon(x_0, \xi) \hat{\rho}_\varepsilon(x_0, \xi)\rangle\big)\\ = \mathrm{Re}\big(\langle\partial_{x_0} a_\varepsilon(x_0, \xi)\hat{w}_\varepsilon(x_0, \xi), \hat{w}_\varepsilon(x_0, \xi)\rangle \big) + 2 \text{Re}\big( \langle a_\varepsilon(x_0, \xi) \hat{\rho}_\varepsilon(x_0, \xi), \hat{w}_\varepsilon(x_0, \xi)\rangle\big)
\end{gather*}
\begin{equation}
\label{eq.3.13}
 \leq |\partial_{x_0} a_\varepsilon| |\hat{w}_\varepsilon(x_0, \xi)|^2 + 2 |a_\varepsilon| |\hat{\rho}_\varepsilon(x_0, \xi)| |\hat{w}_\varepsilon(x_0, \xi)|.  
\end{equation}
In the case where $|\hat{w}_\varepsilon(x_0, \xi)| \geq 1$, taking into account~(\ref{eq3.12}) we obtain from the inequality above that
\begin{align*}
\partial_{x_0} E_\varepsilon(x_0, \xi)&\leq |\partial_{x_0} a_\varepsilon| |\hat{w}_\varepsilon(x_0, \xi)|^2 + 2 |a_\varepsilon| |\hat{\rho}_\varepsilon(x_0, \xi)| |\hat{w}_\varepsilon(x_0, \xi)|\\
 &\leq \big(|\partial_{x_0} a_\varepsilon| + 2 |a_\varepsilon| |\hat{\rho}_\varepsilon(x_0, \xi)| \big) |\hat{w}_\varepsilon(x_0, \xi)|^2 \\ 
 &\leq c (\omega(\varepsilon))^{-L-1} E_\varepsilon(x_0, \xi)
\end{align*}
for some constant $c > 0$. Then, the Gronwall lemma implies that
\[
E_\varepsilon(x_0, \xi) \leq \exp(c (\omega(\varepsilon))^{-L-1} T) E_\varepsilon(0, \xi)
\]
for all $T > 0$. Hence, by~(\ref{eq3.12}) we obtain for the constant $c_1$ independent of $t \in [0, T]$ and $\xi$ that
\[
c_0 |\hat{w}_\varepsilon(x_0, \xi)|^2 \leq E_\varepsilon(x_0, \xi)\leq \exp(c (\omega(\varepsilon))^{-L-1} T) E_\varepsilon(0, \xi) \leq \exp(c_1 (\omega(\varepsilon))^{-L-1} T) |\hat{w}_\varepsilon(0, \xi)|^2.
\]
Choosing $(\omega(\varepsilon))^{-L-1} \approx \log \varepsilon$, we get
\[
|\hat{w}_\varepsilon(x_0, \xi)|^2 \leq c \varepsilon^{-L-1} |\hat{w}_\varepsilon(0, \xi)|^2
\]
for some positive constant $c$ and some $L$. It implies for all $\xi$ and $x_0 \in [0, T]$ that
\[
|\hat{w}_\varepsilon(x_0, \xi)| \equiv 0,
\]
since $|\hat{w}_\varepsilon(0, \xi)| = 0$.

Now let us consider the case $|\hat{w}_\varepsilon(x_0, \xi)| < 1$. Assume that
\[
|\hat{w}_\varepsilon(x_0, \xi)| \geq c (\omega(\varepsilon))^\alpha
\]
for some constant $c$ and $\alpha > 0$, i.e.
\begin{equation}
\label{eq3.15}
\frac{1}{|\hat{w}_\varepsilon(x_0, \xi)|} \leq C (\omega(\varepsilon))^{-\alpha}.
\end{equation}
In this case, from~(\ref{eq3.15}) noting
\[
|\hat{w}_\varepsilon(x_0, \xi)| = \frac{|\hat{w}_\varepsilon(x_0, \xi)|^2}{|\hat{w}_\varepsilon(x_0, \xi)|} \leq C (\omega(\varepsilon))^{-\alpha} |\hat{w}_\varepsilon(x_0, \xi)|^2
\]
and~(\ref{eq3.12}), we get from~(\ref{eq.3.13}) the following energy estimate
\[
\partial_t E_\varepsilon(x_0, \xi) \leq C (\omega(\varepsilon))^{-L_1} E_\varepsilon(x_0, \xi),
\]
where $L_1 = L + \max\{\alpha\}$. Again applying the Gronwall lemma, we arrive at
\[
|\hat{w}_\varepsilon(x_0, \xi)|^2 \leq \exp(C' (\omega(\varepsilon))^{-L_1} T) |\hat{w}_\varepsilon(0, \xi)|^2.
\]
Then, taking $(\omega(\varepsilon))^{-L_1} \approx \log \varepsilon$, it follows that
\[
|\hat{w}_\varepsilon(x_0, \xi)|^2 \leq c' \varepsilon^{-L_1} |\hat{w}_\varepsilon(0, \xi)|^2
\]
for some $c'$ and some (new) $L_1$, which implies
\[
|\hat{w}_\varepsilon(x_0, \xi)| = 0
\]
for all $\xi$ and $x_0 \in [0, T]$, since we have $|\hat{w}_\varepsilon(0, \xi)| = 0$.

The case $|\hat{w}_\varepsilon(x_0, \xi)| \leq c (\omega(\varepsilon))^\alpha$ for some constant $c$ and $\alpha > 0$ is trivial. Thus, the first part is proved.


\vspace{0.5cm}
(ii) We prove this part in the similar way as Part (i) but using the quasi-symmetrisers. We assume that the Cauchy problem has another solution $u \in \mathcal{G}([0, T]; H^{-\infty}_{(s)})$. At the level of representatives this means that
\[
\begin{cases}
\partial_{x_0} (u_\varepsilon - v_\varepsilon)(x_0) + a_\varepsilon(x_0)\mathcal{D}(u_\varepsilon - v_\varepsilon)(x_0) = \rho_\varepsilon(x_0), & x_0 \in [0, T], \\
(u_\varepsilon - v_\varepsilon)(0) = 0,
\end{cases}
\]
where $\rho_\varepsilon$ is $C^\infty([0, T]; H^{-\infty}_{(s)})$-negligible. 
The equation will be studied after the group Fourier transform, as a system of the type
\[
\partial{x_0} \hat{w}_\varepsilon(x_0, \xi) + i \xi a_\varepsilon(x_0, \xi) \hat{w}_\varepsilon(x_0, \xi)= \hat{\rho}_\varepsilon(x_0, \xi),
\]
for all $\xi \in \mathbb{R}^n$.
In this case we can define the energy as
\[
E_\varepsilon(x_0, \xi, \delta) := \mathrm{Re}(\langle (a_\varepsilon(x_0, \xi) + \delta^2) \hat{w}_\varepsilon(x_0, \xi), \hat{w}_\varepsilon(x_0, \xi)\rangle)
\]

Taking into account the properties in the proof of the Theorem~\ref{th1.1} we conclude that the Cauchy problem~(\ref{CP}) has a unique solution $v \in L^2([0, T]; H^{-\infty}_{(s)})$ for all $s \in \mathbb{R}$.\\
This completes the proof of Theorem~\ref{th1.7}. 

\begin{theorem}(Consistency-1)
\label{th1.8}
Let $T > 0$.
\begin{enumerate}
    \item[(i)] Let $a \in L^\infty_1([0, T])$ with $a(x_0) \geq a_0 > 0$. Let $s \in \mathbb{R}$, $u_0 \in H^{s}$ and $f \in C([0, T]; H^s)$. Let $u$ be a very weak solution of $H^s$-type of~(\ref{CP}). Then for any regularising families $a_\varepsilon$ and $f_\varepsilon$ in Definition~\ref{def-1.4-hip}, any representative $(u_\varepsilon)_\varepsilon$ of $u$ converges in $C([0, T]; H^{s})$ as $\varepsilon \to 0$ to the unique classical solution in $C([0, T]; H^{s})$ of the Cauchy problem~(\ref{CP}) given by Theorem~\ref{th1.2}.
    
    \item[(ii)] Let $a \in C^\ell([0, T])$ with $\ell \geq 2$ be such that $a(x_0) \geq 0$. Let $1 \leq s < 1 + \ell/2$ and $u_0\in H^{-\infty}(s)$ and $f \in C([0, T]; H^{-\infty}(s))$. Let $u$ be a very weak solution of $H^{-\infty}(s)$-type of~(\ref{CP}). Then for any regularizing families $a_\varepsilon$ and $f_\varepsilon$ in Definition~\ref{def-1.4-hip}, any representative $(u_\varepsilon)_\varepsilon$ of $u$ converges in $C^2([0, T]; H^{-\infty}(s))$ as $\varepsilon \to 0$ to the unique classical solution in $C^2([0, T]; H^{-\infty}(s))$ of Cauchy problem~(\ref{CP}) given by Theorem~\ref{th1.2}.
\end{enumerate}
\end{theorem}

Similarly, we can show other consistency ``cases'' of Theorem~\ref{th1.8} as following.

\begin{theorem}(Consistency-2)
\label{th1.9}
Let $T > 0$.
\begin{enumerate}
    \item[(i)] Let $a(x_0) \geq a_0 > 0$ and $a \in C^\alpha([0, T])$ with $0 < \alpha < 1$. Let $1 \leq s < 1 + \alpha/(1 - \alpha)$, $u_0 \in H^{-\infty}(s)$ and $f \in C([0, T]; H^{-\infty}(s))$. Let $u$ be a very weak solution of $H^{-\infty}(s)$-type of~(\ref{CP}). Then for any regularising families $a_\varepsilon$ and $f_\varepsilon$ in Definition~\ref{def-1.4-hip}, any representative $(u_\varepsilon)_\varepsilon$ of $u$ converges in $C^2([0, T]; H^{-\infty}(s))$ as $\varepsilon \to 0$ to the unique classical solution in $C^2([0, T]; H^{-\infty}(s))$ of the Cauchy problem~(\ref{CP}) given by Theorem~\ref{th1.2}.
    
    \item[(ii)] Let $a(x_0) \geq 0$ and $a \in C^\alpha([0, T])$ with $0 < \alpha < 2$. Let $1 \leq s < 1 + \alpha/2$, $u_0 \in H^{-\infty}(s)$ and $f \in C([0, T]; H^{-\infty}(s))$. Let $u$ be a very weak solution of $H^{-\infty}(s)$-type of~(\ref{CP}). Then for any regularising families $a_\varepsilon$ and $f_\varepsilon$ in Definition~\ref{def-1.4-hip}, any representative $(u_\varepsilon)_\varepsilon$ of $u$ converges in $C^2([0, T]; H^{-\infty}(s))$ as $\varepsilon \to 0$ to the unique classical solution in $C^2([0, T]; H^{-\infty}(s))$ of the Cauchy problem~(\ref{CP}) given by Theorem~\ref{th1.2}.
\end{enumerate}
\end{theorem}


\section*{Proof of Theorem~\ref{th1.8}}

\subsection*{Comparison of Classical Solution and Very Weak Solution}

Here, we compare the classical solution $\tilde{u}$ given by Theorem~\ref{th1.1} with the very weak solution $u$ provided by Theorem~\ref{th1.8}. By the definition of the classical solution, we have for the classical solution $\tilde{u}$:
\begin{equation}
\begin{cases}
\partial_{x_0} \tilde{u}(x_0) + a(x_0) D \tilde{u}(x_0) = 0, & x_0 \in [0, T], \\
\tilde{u}(0) = u_0 \in L^2(\mathbb{R}^n).
\end{cases}
\label{eq3.18}
\end{equation}

From the definition of the very weak solution $u$, we also know that there exists a representative $(u_\varepsilon)_\varepsilon$ of $u$ such that:
\begin{equation}
\begin{cases}
\partial_{x_0} u_\varepsilon(x_0) + a_\varepsilon(x_0) D u_\varepsilon(x_0) = 0, & x_0 \in [0, T], \\
u_\varepsilon(0) = u_0 \in L^2(\mathbb{R}^n),
\end{cases}
\label{eq3.19}
\end{equation}
for suitable embeddings of $a(x_0)$. Since $(a_\varepsilon - a)_\varepsilon \to 0$ in $C([0, T])$ for $a \in L^\infty_1([0, T])$, then \eqref{eq3.18} becomes:
\begin{equation}
\begin{cases}
\partial_{x_0} \tilde{u}(x_0) + a_\varepsilon(x_0) D \tilde{u}(x_0) = n_\varepsilon(x_0), & x_0 \in [0, T], \\
\tilde{u}(0) = u_0 \in L^2(G),
\end{cases}
\label{eq3.20}
\end{equation}
where $n_\varepsilon(x_0) = (a_\varepsilon(x_0) - a(x_0)) D \tilde{u}(x_0) \in C([0, T]; H^s)$ and converges to $0$ in this space as $\varepsilon \to 0$.

By virtue of \eqref{eq3.19} and \eqref{eq3.20}, we note that $\tilde{u} - u_\varepsilon$ solves the following Cauchy problem:
\begin{equation}
\begin{cases}
\partial_{x_0} (\tilde{u} - u_\varepsilon)(x_0) + a_\varepsilon(x_0) D (\tilde{u} - u_\varepsilon)(x_0) = n_\varepsilon(x_0), & x_0 \in [0, T], \\
(\tilde{u} - u_\varepsilon)(0) = 0, \\
(\partial_{x_0} \tilde{u} - \partial_{x_0} u_\varepsilon)(0) = 0.
\end{cases}
\label{eq:3.21}
\end{equation}

Then, similarly as in the proof of Theorem~\ref{th1.7}, we apply Fourier transform to get the following energy estimate:
\[
\partial_{x_0} E_\varepsilon(x_0, \xi) \leq \|\partial_{x_0} a_\varepsilon(x_0)\| \|(\tilde{V} - V_\varepsilon)(x_0, \xi)\|^2 + 2 \|a_\varepsilon(x_0)\| \|n_\varepsilon(x_0, \xi)\| \|(\tilde{V} - V_\varepsilon)(x_0, \xi)\|,
\]
which implies:
\[
\partial_{x_0} E_\varepsilon(x_0, \xi) \leq c_1 \|(\tilde{V} - V_\varepsilon)(x_0, \xi)\|^2 + c_2 \|n_\varepsilon(x_0, \xi)\| \|(\tilde{V} - V_\varepsilon)(x_0, \xi)\|,
\]
since the coefficient $a_\varepsilon(x_0)$ is regular enough. Then, noting $\|(\tilde{V} - V_\varepsilon)(0, \xi)\| = 0$ and $n_\varepsilon \to 0$ in $C([0, T]; H^s)$, and continuing to discuss as in Theorem~\ref{th1.7}, we arrive at:
\[
\|(\tilde{V} - V_\varepsilon)(x_0, \xi)\| \leq c (\omega(\varepsilon))^\alpha,
\]
for some positive constants $c$ and $\alpha$, which concludes that:
\[
u_\varepsilon \to \tilde{u} \quad \text{in } C([0, T]; H^s) .
\]
Furthermore, since any other representative of $u$ will differ from $(u_\varepsilon)_\varepsilon$ by a $C^\infty([0, T]; H^s)$-negligible net, the limit is the same for any representative of $u$.

\subsection*{(ii) Proof of Part (ii)}

Part (ii) can be proven as Part (i) with slight modifications.

This completes the proof of Theorem~\ref{th1.8}. $\Box$



%%%% KIRA FROM HERE%%%%%%%%
 \hrulefill

 Adapting the results from the paper, firstly the case where the right hand side of the equation is null.
 \[
\begin{cases}
    v_t(x_0) - i\beta a(x_0) v(x_0) = 0\\
    v(0) = v_0
\end{cases}
\]


Case 1: $a \in Lip([0,T])$, $a(x_0) \geq a_0 > 0$. This is the simplest case that can be
treated by a classical argument. 

Thus we define the energy as
\[E(x_0) :=(a(x_0)v(x_0), v(x_0))\]

In this case the continuity of $a(x_0)$ ensures the existence of two strictly
positive constants $a_0$ and $a_1$ such that:
\[a_0 = \min_{t\in[0,T]}a(x_0) \qquad and \qquad a_1 = \max_{t\in[0,T]}a(x_0).\]

A straightforward calculation to estimate its variations in time yields
the following inequality that will help us to get such estimate:
\[a_0\norm{v(x_0)}^2 \leq E(x_0) \leq a_1\norm{v(x_0)}^2 .\]
A straightforward calculation, together with (3.6), gives the following estimate:

\begin{align*}
    E_t(x_0)  &=(a_t(x_0)v (x_0), v (x_0)) + (a(x_0)v_t (x_0), v (x_0)) + (a(x_0)v (x_0), v_t (x_0))\\
            &=(a_t(x_0)v (x_0), v (x_0)) + i\beta(a(x_0)a(x_0)v(x_0), v (x_0)) - i\beta(a(x_0)v (x_0), a(x_0)v (x_0))\\
            &=(a_t(x_0)v (x_0), v (x_0))\\
            &=\norm{a_t(x_0)}\norm{v(x_0)}^2
\end{align*}

thus setting 
\[c' := a_0^{-1}\sup_{t\in[0,T]}\norm{a_t(x_0)}\]
we get from (3.7) using (3.6) that
\[E_t(x_0) \leq c'E(x_0).\]
Applying the Gronwall lemma to (3.8), we deduce that there exists a constant $c > 0$ independent of $t \in [0, T]$ such that:
\[E(x_0) \leq cE(0).\]
Therefore, putting together (3.9) and (3.6) we obtain
\[a_0\norm{v (x_0)}^2 \leq E(x_0) \leq cE(0) \leq ca_1\norm{v (0)}^2 = ca_1\norm{v_0}^2\]

We can then rephrase this, asserting that there exists a constant $C > 0$ independent
of t such that 
\[\norm{v (x_0)}^2 \leq C\norm{v_0}^2.\]

Now fourrier transform
\[
\begin{cases}
\hat{u}_t(t, \x) -i\xi a(x_0)\hat{u}(x_0, \x) = 0\\
\hat{u}(0, \x) = \hat{u}_0(\x)
\end{cases}
\]

this is the same as above so we get:
\[\norm{\hat{u}(x_0, \x)}^2 \leq C\norm{\hat{u}_0(\x)}^2\]

\hrulefill

Now for the case $f\neq0$
 \[
\begin{cases}
    v_t(x_0) - i\beta a(x_0) v(x_0) = f(x_0)\\
    v(0) = v_0
\end{cases}
\]


Case 1: $a \in Lip([0,T])$, $a(x_0) \geq a_0 > 0$. This is the simplest case that can be
treated by a classical argument. 

Thus we define the energy as
\[E(x_0) :=(a(x_0)v(x_0), v(x_0))\]

In this case the continuity of $a(x_0)$ ensures the existence of two strictly
positive constants $a_0$ and $a_1$ such that:
\[a_0 = \min_{t\in[0,T]}a(x_0) \qquad and \qquad a_1 = \max_{t\in[0,T]}a(x_0).\]

A straightforward calculation to estimate its variations in time yields
the following inequality that will help us to get such estimate:
\[a_0\norm{v(x_0)}^2 \leq E(x_0) \leq a_1\norm{v(x_0)}^2 .\]
A straightforward calculation, together with (3.6), gives the following estimate:
\begin{align*}
    E_t(x_0)  &=(a_t(x_0)v (x_0), v (x_0)) + (a(x_0)v_t (x_0), v (x_0)) + (a(x_0)v (x_0), v_t (x_0))\\
            &=(a_t(x_0)v (x_0), v (x_0)) + [a(x_0)(i\beta a(x_0)v(x_0) + f(x_0)), v (x_0)] + [a(x_0)v (x_0), (i\beta a(x_0)v(x_0) + f(x_0))]\\
            &=(a_t(x_0)v (x_0), v (x_0)) + (a(x_0)f (x_0), v(x_0)) + (a(x_0)v(x_0), f(x_0))\\
            &=\norm{a_t(x_0)}(v (x_0), v (x_0)) +2a(x_0)Re(f(x_0), v(x_0))\\
            &\leq(\norm{a_t(x_0)} + 1)\norm{v(x_0)}^2+a^2(x_0)\norm{f(x_0)}^2\\
            &\leq \sup_{t \in [0,T]}\{\norm{a_t(x_0)} + 1\} \norm{v(x_0)} +a^2(x_0)\norm{f(x_0)}^2\\
            &\leq C_1E(x_0) + C_2\norm{f(x_0)}^2
\end{align*}

for some positive constants C1 and C2. Applying Gronwall’s lemma and noting (4.7),
we get

\[\norm{v (x_0)}^2 \leq a_0^{-1}E(x_0) \leq C_1\norm{v (0)}^2 +  C_2\sup_{t \in [0,T]}\norm{f(x_0)}^2\]

Now fourrier transform
\[
\begin{cases}
\hat{u}_t(t, \x) -i\xi a(x_0)\hat{u}(x_0, \x) = \hat{f}\\
\hat{u}(0, \x) = \hat{u}_0(\x)
\end{cases}
\]

this is the same as above so we get:
\[\norm{\hat{u}(x_0, \x)}^2 \leq C_1\norm{\hat{u}_0(\x)}^2 + C_2\norm{\hat{f}(\x)}^2\]

\hrulefill
%%%%


\begin{thebibliography}{2025}

\bibitem{BDS} F. Brackx, R. Delanghe, and F. Sommen, Clifford Analysis, Pitman-Longman, 1982.

\bibitem{DSS}
R. Delanghe, F. Sommen and V. Sou\u{c}ek, \textit{Clifford algebras and spinor-valued functions. A function theory for the Dirac operator}, Mathematics and its Applications-Vol.53, Kluwer Academic Publishers,
Dordrecht etc., 1992.


\bibitem{HSOE-GarettoRuz}
Garetto, C., Ruzhansky, M. Hyperbolic Second Order Equations with Non-Regular Time Dependent Coefficients. Arch Rational Mech Anal 217, 113–154 (2015). https://doi.org/10.1007/s00205-014-0830-1

\bibitem{GS} K. G\"urlebeck; W. Spr\"ossig. \textit{Quaternionic and Clifford Calculus for Physicists and Engineers}, Wiley and Sons Publ., 1997.	


\bibitem{VWS-YessinkegenovRuz}
Ruzhansky, M.,  Yessirkegenov, N.  Very weak solutions to hypoelliptic wave equations. J. Differential Equations, 268 (2020), no. 5, 2063–2088.
\end{thebibliography}


\end{document}